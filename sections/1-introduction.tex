\section{Introduction}
\label{sec:introduction}
\section{Introduction}
\label{sec:introduction}
The multiple sequence alignment (MSA) problem has been extensively studied, leading to the development of various approaches over the years. With the rapid growth of Big Data, the demand for faster and more accurate MSA methods has increased significantly. A fast and accurate MSA approach would have many implications in the field of biology. Possible efficient solutions to this problem can provide an essential tool to help biologists research and solve many pressing issues in their field. This approach could help biologists track evolution processes and better determine the relationships between different species.\\
\\Many recent MSA algorithms focus on improving computational efficiency while minimizing alignment error. However, achieving both speed and accuracy remains a challenge, as more accurate alignments often require increased complexity, resulting in longer run times. The ongoing goal in MSA research is to balance these two aspects—developing methods that compute fast yet produce high-quality alignments. \\
\\There are two main approaches that have been used to construct multiple sequence alignment throughout the different approaches that have come out over the recent years. One, the star alignment strategy is a method that chooses one sequence from the entire data set of sequences and uses it as the center for the MSA. This can be problematic as it can be impossible to choose one sequence that can be relatively close to the rest. For example, consider a dataset with two outliers and the rest of the data points lying within a very small area. The outlier might have to be the center as it has the lowest average distance from the others. This can make a very loose multiple sequence alignment as all the pairwise alignments can be made from sequences with very large distances between each other. Another approach is the progressive alignment strategy that uses a pre-built guide tree to build the MSA. This method starts at every position of the tree where two leaf nodes are connected together and aligns them as a pairwise alignment. Then, from all the leaf nodes, it moves up and combines two trees together based on the alignment of two sequences, each from different trees, that most related to each other or adds a leaf to the already existing tree based on the tree node it is most related to. [5] \\
\\One influential approach, Tandy Warnow's Pasta [1], has served as a benchmark for many recent methods. Pasta, along with our previous work on CLAM, provided a foundation for developing the approach presented in this paper. To address this challenge, we introduce **name of method**, which builds on the tree implementation from CLAM [2]. Our approach employs hierarchical clustering to construct a guide tree, enabling a bottom-up alignment process. It uses both the star and progressive alignment strategies within the different stages to allow for smaller edit distances and better time performances. This design aims to offer a solution that maintains fast computation without compromising alignment quality. 
\cite{mirarab2015astral}
