\documentclass{article}
\pdfoutput=1
%%
%% \BibTeX command to typeset BibTeX logo in the docs
\AtBeginDocument{%
  \providecommand\BibTeX{{%
    Bib\TeX}}}
% The preceding line is only needed to identify funding in the first footnote. If that is unneeded, please comment it out.
\usepackage{cite}
\usepackage{amsmath,amssymb,amsfonts}
\usepackage{algorithm}
\usepackage{algpseudocode}
% \usepackage{algpseudocodex}
\usepackage{graphicx}
\usepackage{textcomp}
\usepackage{xcolor}
\usepackage{subcaption}
\usepackage{multirow}
\newcommand{\rulesep}{\unskip\ \vrule\ }

\DeclareMathOperator*{\argmax}{arg\,max}
\DeclareMathOperator*{\argmin}{arg\,min}

\begin{document}

\title{Title for MSA paper}

\author{
    Emily Light \\
    Department of Computer Science and Statistics\\
    University of Rhode Island\\
    Kingston, RI\\
    \texttt{emily\_light@uri.edu} \\
    \And
    Noah M. Daniels \\
    Department of Computer Science and Statistics\\
    University of Rhode Island\\
    Kingston, RI\\
    \texttt{noah\_daniels@uri.edu} \\
    \And
    Najib Ishaq \\
    Department of Computer Science and Statistics\\
    University of Rhode Island\\
    Kingston, RI\\
    \texttt{najib\_ishaq@zoho.com} \\
}

\maketitle

\begin{abstract}
        The multiple sequence alignment problem has been investigated and many approaches have been developed over recent years. With the explosion of Big Data, the demand for fast and accurate multiple sequence approaches has drastically increased. Tandy Warnow's Pasta approach [1] served as a baseline for many more recent approaches. In addition to the other approaches, Pasta served as our reference point for the approach we developed using the previous work of CLAM. Our approach uses hierarchical clustering to form a guide tree that allows us to form a multiple sequence alignment using a bottom-up approach. \textbf{add sentences about results}
\end{abstract}

\section{Introduction}
\label{sec:introduction}
\section{Introduction}
\label{sec:introduction}
The multiple sequence alignment (MSA) problem has been extensively studied, leading to the development of various approaches over the years. With the rapid growth of Big Data, the demand for faster and more accurate MSA methods has increased significantly. A fast and accurate MSA approach would have many implications in the field of biology. Possible efficient solutions to this problem can provide an essential tool to help biologists research and solve many pressing issues in their field. This approach could help biologists track evolution processes and better determine the relationships between different species.\\
\\Many recent MSA algorithms focus on improving computational efficiency while minimizing alignment error. However, achieving both speed and accuracy remains a challenge, as more accurate alignments often require increased complexity, resulting in longer run times. The ongoing goal in MSA research is to balance these two aspects—developing methods that compute fast yet produce high-quality alignments. \\
\\There are two main approaches that have been used to construct multiple sequence alignment throughout the different approaches that have come out over the recent years. One, the star alignment strategy is a method that chooses one sequence from the entire data set of sequences and uses it as the center for the MSA. This can be problematic as it can be impossible to choose one sequence that can be relatively close to the rest. For example, consider a dataset with two outliers and the rest of the data points lying within a very small area. The outlier might have to be the center as it has the lowest average distance from the others. This can make a very loose multiple sequence alignment as all the pairwise alignments can be made from sequences with very large distances between each other. Another approach is the progressive alignment strategy that uses a pre-built guide tree to build the MSA. This method starts at every position of the tree where two leaf nodes are connected together and aligns them as a pairwise alignment. Then, from all the leaf nodes, it moves up and combines two trees together based on the alignment of two sequences, each from different trees, that most related to each other or adds a leaf to the already existing tree based on the tree node it is most related to. [5] \\
\\One influential approach, Tandy Warnow's Pasta [1], has served as a benchmark for many recent methods. Pasta, along with our previous work on CLAM, provided a foundation for developing the approach presented in this paper. To address this challenge, we introduce **name of method**, which builds on the tree implementation from CLAM [2]. Our approach employs hierarchical clustering to construct a guide tree, enabling a bottom-up alignment process. It uses both the star and progressive alignment strategies within the different stages to allow for smaller edit distances and better time performances. This design aims to offer a solution that maintains fast computation without compromising alignment quality. 
\cite{mirarab2015astral}

\section{Methods}
\label{sec:methods}
\subsection{Guide Tree}
To create our Multiple Sequence Alignment (MSA), we performed multiple pairwise alignments, combining each one iteratively to build a cohesive MSA. This process was guided by a tree structure generated from a previous algorithm, CHESS, which uses Entropy-Scaling Search [3]. The guide tree applied the Levenshtein distance metric to cluster similar sequences closely together, optimizing the alignment order by aligning the most similar sequences first. This approach strengthened the final MSA by helping the algorithm determine the optimal sequence pairings and alignment order, ultimately enhancing the accuracy and consistency of the alignment.
\subsection{Clustering}
It uses a clustering approach to make the task of calculating a large MSA into smaller easier tasks to complete. This approach allows our MSA to be efficient without sacrificing a significant amount of quality from the MSA. Our clustering approach is a hierarchal approach, used previously for CHESS [3]. After defining these clusters, we can examine each cluster individually. As discussed in CHESS, each cluster contains a center data point as well as serval other data points at a relatively close distance to the center. The center of each cluster is chosen as the point that has the lowest average distance between the rest of the points within the cluster as this results in better pairwise alignments. For this specific implementation, the distance of the points are based on the Needleman-Wunsch sequence alignment algorithm. [4]. Once all the clusters and the centers are defined, the algorithm begins completing pairwise alignments. This algorithm starts by examining each cluster separately. For each cluster, it uses the center to align all the other sequences. Therefore, for a cluster of size N, there will be N - 1 pairwise alignments and N - 1 different alignments of the center. With these pairwise alignments, our algorithm can go into the next section that discusses the method we use to merge all these pairwise alignments into one MSA.
\subsection{Merging}
Within this approach, there is two stages of the merging process that allows us to convert the large amount of pairwise sequence alignments to one, large multiple sequence alignment. This approach allows us to use information from all the pairwise alignments completed in the previous steps that we have discussed above. The first stage of this process focuses on creating very small MSAs out of the individual clusters that the dataset was divided into. The second stage uses the centers to As explained earlier, each cluster has a center that had been aligned separately with the rest of the points in the cluster. As a result of this, there is N - 1 different versions of an aligned version of the center. We use these versions to create one center that aligns with all the non-center sequences. \\
\\We merge all the centers together by taking one center sequence and adding all the gaps of the different versions to one center. We start with just the first pairwise alignment and add it to the empty aligned cluster. Next, we examine the next pairwise alignment. This alignment will share the same center but will have a different non-center sequence. Because of this, it is very likely that the two versions of the aligned center will have gaps in different positions. The goal is to have one version of the center that aligns to all of the other sequences. To ensure that we have a center that aligns to the two other sequences we are considering, we must look at the differences between the two versions of the center. When we see a difference between the two centers, it will either be because of a gap was inserted in the first version or second of the center. To get the centers to match, we must insert a gap in the correct position to both the center missing the gap and its corresponding non-center sequence. After going through all the gaps in both versions of the center using this method, there will be one common center that aligns to both non-center sequences. This method can be repeated to add the rest of the non-center sequences and to create one fully aligned cluster. This merging starts at each of the bottom-most edges of the cluster tree and works its way up to the root, resulting in all the sequences within the data set to be aligned to each other.

\vspace{0.5cm} 

\begin{algorithm}[H]
\caption{Merge Gaps}
\begin{algorithmic}[1]
\For{$gap$ \textbf{in} $pairwise\_center$ \textbf{or} $cluster\_center$}
    \If{$gap$ \textbf{in} $pairwise\_center$ \textbf{and} $gap$ \textbf{not in} $cluster\_center$}
        \For{$sequence$ \textbf{in} $cluster$}
            \State Insert $gap$ into $sequence$ at $position$
        \EndFor
    \ElsIf{$gap$ \textbf{not in} $pairwise\_center$ \textbf{and} $gap$ \textbf{in} $cluster\_center$}
        \State Insert $gap$ into $pairwise\_center$ at $position$
        \State Insert $gap$ into $pairwise\_seq$ at $position$
    \EndIf
\EndFor
\State \textbf{Return} $clusters$
\end{algorithmic}
\end{algorithm}

\section{Datasets And Benchmarking}
\label{sec:datasets-and-benchmarks}

\section{Results}
\label{sec:results}

\section{Discussion and Future Work}
\label{sec:discussion-and-future-work}


\bibliographystyle{IEEEtran}
\bibliography{references}

\end{document}
